\documentclass{beamer}
%\includeonlyframes{current}

% This file is a solution template for:

% - Talk at a conference/colloquium.
% - Talk length is about 20min.
% - Style is ornate.
%
% Copyright 2004 by Till Tantau <tantau@users.sourceforge.net>.
%
% In principle, this file can be redistributed and/or modified under
% the terms of the GNU Public License, version 2.
%
% However, this file is supposed to be a template to be modified
% for your own needs. For this reason, if you use this file as a
% template and not specifically distribute it as part of a another
% package/program, I grant the extra permission to freely copy and
% modify this file as you see fit and even to delete this copyright
% notice. 


\mode<presentation>
{
  \usetheme{Warsaw}
  % or ...

  \setbeamercovered{transparent}
  % or whatever (possibly just delete it)
}


\usepackage[english]{babel}
% or whatever

\usepackage[latin1]{inputenc}
% or whatever

\usepackage{times, hyperref, graphicx}
\usepackage[T1]{fontenc}
% Or whatever. Note that the encoding and the font should match. If T1
% does not look nice, try deleting the line with the fontenc.


\title[useR! 2013] % (optional, use only with long paper titles)
{An update on useR! 2013 conference}

%\subtitle
%{Include Only If Paper Has a Subtitle}

\author % (optional, use only with lots of authors)
{Vik Gopal}
% - Give the names in the same order as the appear in the paper.
% - Use the \inst{?} command only if the authors have different
%   affiliation.

%\institute[Universities of Somewhere and Elsewhere] % (optional, but mostly needed)
%{
%  \inst{1}%
%  Department of Computer Science\\
%  University of Somewhere
%  \and
%  \inst{2}%
%  Department of Theoretical Philosophy\\
%  University of Elsewhere}
%% - Use the \inst command only if there are several affiliations.
%% - Keep it simple, no one is interested in your street address.

\date % (optional, should be abbreviation of conference name)
{RUGS\\Sep 26 2013}
% - Either use conference name or its abbreviation.
% - Not really informative to the audience, more for people (including
%   yourself) who are reading the slides online

%\subject{Theoretical Computer Science}
% This is only inserted into the PDF information catalog. Can be left
% out. 



% If you have a file called "university-logo-filename.xxx", where xxx
% is a graphic format that can be processed by latex or pdflatex,
% resp., then you can add a logo as follows:

% \pgfdeclareimage[height=0.5cm]{university-logo}{university-logo-filename}
% \logo{\pgfuseimage{university-logo}}



% Delete this, if you do not want the table of contents to pop up at
% the beginning of each subsection:
\AtBeginSubsection[]
{
  \begin{frame}<beamer>
    \frametitle{Outline}
    \tableofcontents[currentsection,currentsubsection]
  \end{frame}
}


% If you wish to uncover everything in a step-wise fashion, uncomment
% the following command: 

%\beamerdefaultoverlayspecification{<+->}


\begin{document}

\begin{frame}
  \titlepage
\end{frame}

\begin{frame}
  \frametitle{Outline}
  \tableofcontents
  % You might wish to add the option [pausesections]
\end{frame}


% Structuring a talk is a difficult task and the following structure
% may not be suitable. Here are some rules that apply for this
% solution: 

% - Exactly two or three sections (other than the summary).
% - At *most* three subsections per section.
% - Talk about 30s to 2min per frame. So there should be between about
%   15 and 30 frames, all told.

% - A conference audience is likely to know very little of what you
%   are going to talk about. So *simplify*!
% - In a 20min talk, getting the main ideas across is hard
%   enough. Leave out details, even if it means being less precise than
%   you think necessary.
% - If you omit details that are vital to the proof/implementation,
%   just say so once. Everybody will be happy with that.

\section{About the conference}
\begin{frame}[t]
  \frametitle{useR! 2013}
\begin{itemize}
  \item Conference was held from July 10 - 12, 2013 in Albacete, Spain at
    University of Castilla-La Mancha.
\end{itemize}
%URL is \url{http://www.edii.uclm.es/~useR-2013/}
\begin{center}
\only<1>{\includegraphics[scale=0.3]{figs/groupl.jpg}}
\only<2>{\includegraphics[scale=0.3]{figs/export/1-IMG_0015}
\hskip3pt
\includegraphics[scale=0.3]{figs/export/1-IMG_0021}}
\only<3>{\includegraphics[scale=0.3]{figs/export/1-IMG_0028}
\hskip3pt
\includegraphics[scale=0.3]{figs/export/1-IMG_0032}}
\end{center}
\begin{center}
\only<3>{(Conference URL at the end of this presentation.)}
\end{center}
\end{frame}

\section{Interesting companies}
\begin{frame}[t,label=current]
  \frametitle{Tibco}
  Tibco is the company that bought over Insightful. With Tibco's Spotfire
  sofware, you can:
  \begin{itemize}
  \item Get access to a broad array of predictive analytic and cutting-edge
    statistical analysis tools.
  \item Develop analytics applications as much as five times faster.
  \item Extend these tools with functions from a contributed package 
    \href{http://csan.insightful.com}{CSAN repository}.
  \only<2->{\item Also working on a new implementation of the S language, that
    is focused on improving the memory management issues of R.}
  \only<2->{\item What defines the S language?
    \begin{itemize}
      \item Brown (1984) $\rightarrow$ Blue (1988) $\rightarrow$ White (1992) 
        $\rightarrow$ Green (1998).
    \end{itemize}}
%S was first introduced by Becker and Chambers (1984) in what's known as the
%'brown' book. The new S language was described by Becker, Chambers and Wilks
%(1988) in the 'blue' book. Chambers and Hastie (1992) edited a book discussing
%statistical modeling in S, called the 'white' book. The latest version of the S
%language is described by Chambers (1998) in the 'green' book, but R is largely
%an implementation of the versions documented in the blue and white books.
%Chamber's (2008) latest book focuses on Programming with R.
% MENTION LEXICAL SCOPING.
  \end{itemize}
\end{frame}

\begin{frame}
  \frametitle{Mango Solutions}
  \begin{itemize}
   \item A UK based company, started in 2002.
   \item They provide the following services:
    \begin{itemize}
      \item Consulting 
      \item Training, and 
      \item R validation.
    \end{itemize}
  \end{itemize}
  \pause
  \begin{block}{What I understand from this:}
    \begin{itemize}
      \item More people \emph{want} to use the analysis tools within R.
      \item Like Android, it is getting difficult to maintain compatibility
        across versions of R and of packages.
    \end{itemize}
  \end{block}
\end{frame}

\section{Interesting concepts}
\begin{frame}[fragile,label=current]
  \frametitle{R version 3.0.0 - what's new?}
  \begin{itemize}
    \item Duncan Murdoch gave a talk on the latest version of R.
    \item What does R x.y.z actually mean? \pause
    \item Support for numeric indices $2^{31}$ and larger. So now
    \begin{verbatim}
    > x[2^31] <- y
    \end{verbatim}
    will work on 64-bit machines, extending the x-vector if necessary. \pause
    \item Bounds checking when calling compiled code. \pause
    \item Vignettes can now be written with engines other than Sweave. Any
      engine that processes R code + documentation to create a latex file can 
      be used, e.g. knitr. See this
\href{http://cran.r-project.org/doc/manuals/R-exts.html#Non_002dSweave-vignettes}{page} 
      in the Writing R Extensions section. \pause
    \item What exactly has helped in the dramatic increase in R users? \pause
    % framework for sharing code, repeated answers by Duncan.
  \end{itemize}
\end{frame}

\begin{frame}[label=current]
  \frametitle{BigR Data}
  \begin{itemize}
    \item Hadley Wickham gave a talk on handling bigger data in R
    \item The full talk can be found here: \url{http://bit.ly/bigrdata2}
    \item ``Visualisation reveals the unexpected but does not scale well; models
      scale well, but they are so precise that they seldom reveal new things''
    \item Who doesn't use at least one of reshape2, plyr or ggplot2?
    \item Two new packages for handling big big data:
      \begin{enumerate}
        \item dplyr
        \item bigvis
      \end{enumerate}
  \end{itemize}
\end{frame}

\begin{frame}
  \frametitle{MCMC or INLA?}
  \begin{itemize}
    \item Havard Rue gave a talk on R-INLA
    \item Steve Scott gave a talk on BOOM
    % mention Google correlate
    \item Both are methods for carrying out Bayesian inference.
    \item Although the latter is more general, I am keen to learn about INLA.
  \end{itemize}
\end{frame}

\begin{frame}
  \frametitle{Naming conventions in R}
  \begin{itemize}
    \item A humorous talk on the inconsistencies of R users!
    \item Even base R does not one of the following consistently:
      \begin{itemize}
      \item lowerCamelCase
      \item upperCamelCase
      \item alllowercase
      \item period.separated 
      \item underscore\_separated
      \end{itemize}
    \item Read more about it
\href{http://journal.r-project.org/archive/2012-2/RJournal_2012-2_Baaaath.pdf}
{here}.
  \end{itemize}
\end{frame}

\section{Interesting packages}
\begin{frame}
  \frametitle{data.table Package}
  \begin{itemize}
    \item data.table can be thought of as providing an enhanced data frame.
    \item When used correctly, it can be much faster than a data.frame
    \item The main features are:
      \begin{itemize}
        \item Use of keys for indexing rows. 
        \item Fast grouping
        \item Fast time series joins.
      \end{itemize}
  \end{itemize}
\end{frame}

\begin{frame}
  \frametitle{Must learn packages (from my p.o.v, of course}
  \begin{itemize}
    \item Shiny!
    \item knitr (and markdown)
    \item ??
  \end{itemize}
\end{frame}

\section*{Summary}
\begin{frame}
More info:
\begin{itemize}
 \item URL for this year's conference: \url{http://www.edii.uclm.es/~useR-2013/}
 \item URL for next year's conference: \url{http://user2014.stat.ucla.edu/}
 \item TeX file (rugs\_useR2013.tex) for this presentation: On github at singator/rugs
\end{itemize}
\end{frame}
%\section{Motivation}
%
%\subsection{The Basic Problem That We Studied}
%
%\begin{frame}
%  \frametitle{Make Titles Informative. Use Uppercase Letters.}
%  \framesubtitle{Subtitles are optional.}
%  % - A title should summarize the slide in an understandable fashion
%  %   for anyone how does not follow everything on the slide itself.
%
%  \begin{itemize}
%  \item
%    Use \texttt{itemize} a lot.
%  \item
%    Use very short sentences or short phrases.
%  \end{itemize}
%\end{frame}
%
%\begin{frame}
%  \frametitle{Make Titles Informative.}
%
%  You can create overlays\dots
%  \begin{itemize}
%  \item using the \texttt{pause} command:
%    \begin{itemize}
%    \item
%      First item.
%      \pause
%    \item    
%      Second item.
%    \end{itemize}
%  \item
%    using overlay specifications:
%    \begin{itemize}
%    \item<3->
%      First item.
%    \item<4->
%      Second item.
%    \end{itemize}
%  \item
%    using the general \texttt{uncover} command:
%    \begin{itemize}
%      \uncover<5->{\item
%        First item.}
%      \uncover<6->{\item
%        Second item.}
%    \end{itemize}
%  \end{itemize}
%\end{frame}
%
%
%\subsection{Previous Work}
%
%\begin{frame}
%  \frametitle{Make Titles Informative.}
%\end{frame}
%
%\begin{frame}
%  \frametitle{Make Titles Informative.}
%\end{frame}
%
%
%
%\section{Our Results/Contribution}
%
%\subsection{Main Results}
%
%\begin{frame}
%  \frametitle{Make Titles Informative.}
%\end{frame}
%
%\begin{frame}
%  \frametitle{Make Titles Informative.}
%\end{frame}
%
%\begin{frame}
%  \frametitle{Make Titles Informative.}
%\end{frame}
%
%
%\subsection{Basic Ideas for Proofs/Implementation}
%
%\begin{frame}
%  \frametitle{Make Titles Informative.}
%\end{frame}
%
%\begin{frame}
%  \frametitle{Make Titles Informative.}
%\end{frame}
%
%\begin{frame}
%  \frametitle{Make Titles Informative.}
%\end{frame}
%
%
%
%\section*{Summary}
%
%\begin{frame}
%  \frametitle<presentation>{Summary}
%
%  % Keep the summary *very short*.
%  \begin{itemize}
%  \item
%    The \alert{first main message} of your talk in one or two lines.
%  \item
%    The \alert{second main message} of your talk in one or two lines.
%  \item
%    Perhaps a \alert{third message}, but not more than that.
%  \end{itemize}
%  
%  % The following outlook is optional.
%  \vskip0pt plus.5fill
%  \begin{itemize}
%  \item
%    Outlook
%    \begin{itemize}
%    \item
%      Something you haven't solved.
%    \item
%      Something else you haven't solved.
%    \end{itemize}
%  \end{itemize}
%\end{frame}
%
%
%
%% All of the following is optional and typically not needed. 
%\appendix
%\section<presentation>*{\appendixname}
%\subsection<presentation>*{For Further Reading}
%
%\begin{frame}[allowframebreaks]
%  \frametitle<presentation>{For Further Reading}
%    
%  \begin{thebibliography}{10}
%    
%  \beamertemplatebookbibitems
%  % Start with overview books.
%
%  \bibitem{Author1990}
%    A.~Author.
%    \newblock {\em Handbook of Everything}.
%    \newblock Some Press, 1990.
% 
%    
%  \beamertemplatearticlebibitems
%  % Followed by interesting articles. Keep the list short. 
%
%  \bibitem{Someone2000}
%    S.~Someone.
%    \newblock On this and that.
%    \newblock {\em Journal of This and That}, 2(1):50--100,
%    2000.
%  \end{thebibliography}
%\end{frame}

\end{document}
